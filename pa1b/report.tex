\documentclass[a4paper]{article}
\usepackage{latexsym,amssymb,amsmath,amsbsy,amsopn,amstext,xcolor,multicol}
\usepackage{ctex,hyperref,graphicx,wrapfig,fancybox,listings}
\usepackage{pgf,pgfarrows,pgfnodes,pgfautomata,pgfheaps,pgfshade}
\usepackage[top=1in, bottom=1in, left=1.25in, right=1.25in]{geometry}
\graphicspath{{pic/}}
\title{\bf Decaf PA1-B}
\date{2018.11}
\author{翁家翌~2016011446}
\begin{document}
\kaishu
\ttfamily
\maketitle
%\tableofcontents
%\newpage
\section{}
本阶段工作主要如下:
\begin{itemize}
	\item 在\uline{Parser.java}的\uline{parse}函数中添加错误处理方法,按照实验指导的方法进行添加。具体为添加两个新的集合“beginA”和“endA”,然后在开始的时候先判断是否要抛异常,如果是则跳过一些字符,剩下代码与原来一致。此外由于递归分析某条分支上报错,那么整个祖先分支都要报错,因此定义了一个全局变量\uline{hasError}来更方便地处理这件事情。
	\item 在\uline{Parser.spec}中加入LL(1)文法,除\uline{\%\%}、\uline{++}、取子数组、Default和Foreach任务之外剩下与PA1-A基本一致。在通读\uline{Parser.spec}代码之后发现优先级能够通过逐层递归分析得到解决,很受启发。
	\item 在\uline{Tree.java}中加入生成相关文法的AST的类与函数,与PA1-A类似。我现在来看之前写的代码有很多冗余,因此做了一定程度上的修改。
\end{itemize}
\section{}
由于else语句可以为空,因此在文法G[S]:
\begin{lstlisting}[basicstyle=\ttfamily\small, numbers=left, frame=shadowbox, rulesepcolor=\color{red!20!green!20!blue!20},extendedchars=true,breaklines=true]
S -> if C then S E
E -> else S | <empty>
\end{lstlisting}
有
\begin{lstlisting}[basicstyle=\ttfamily\small, numbers=left, frame=shadowbox, rulesepcolor=\color{red!20!green!20!blue!20},extendedchars=true,breaklines=true]
PS(E -> else S) & PS(E -> <empty>) = {else}
\end{lstlisting}
因此Decaf不是严格的LL(1)语言。但是可以对不同的产生式赋予不同的优先级,比如优先产生else,这样一来PS集合的交即为空,能够处理成LL(1)语言。

比如\uline{test2.decaf}中第21-23行:
\begin{lstlisting}[basicstyle=\ttfamily\small, numbers=left, frame=shadowbox, rulesepcolor=\color{red!20!green!20!blue!20},extendedchars=true,breaklines=true]
if (10 < n) {
    n = 10;
}
\end{lstlisting}
此时{PS(E -> else S) = \{else\}},然而{else $\not\in$ PS(E -> <empty>)},因此匹配后面那条规则,冲突得以解决。
\section{}
原始编译结果如下所示:
\begin{lstlisting}[basicstyle=\ttfamily\small, numbers=left, frame=shadowbox, rulesepcolor=\color{red!20!green!20!blue!20},extendedchars=true,breaklines=true]
1 table generation:
     [java] Warning: conflict productions at line 964:
     [java] ElseClause -> ELSE Stmt
     [java] ElseClause -> <empty>
     [java] Parser is successfully generated and written to "Table.java"
\end{lstlisting}
将\uline{[|}和\uline{|]}改成\uline{[}和\uline{]}之后,编译结果如下所示:
\begin{lstlisting}[basicstyle=\ttfamily\small, numbers=left, frame=shadowbox, rulesepcolor=\color{red!20!green!20!blue!20},extendedchars=true,breaklines=true]
1 table generation:
     [java] Warning: conflict productions at line 777:
     [java] Expr9 -> Constant
     [java] Expr9 -> READ_INTEGER '(' ')'
     [java] Expr9 -> READ_LINE '(' ')'
     [java] Expr9 -> THIS
     [java] Expr9 -> NEW AfterNewExpr
     [java] Expr9 -> INSTANCEOF '(' Expr ',' IDENTIFIER ')'
     [java] Expr9 -> '(' AfterParenExpr
     [java] Expr9 -> IDENTIFIER AfterIdentExpr
     [java] Expr9 -> '[' Expr FOR IDENTIFIER IN Expr IfBoolExpr ']'
     [java] Warning: conflict productions at line 964:
     [java] ElseClause -> ELSE Stmt
     [java] ElseClause -> <empty>
     [java] Warning: unreachable production:
     [java] Expr9 -> '[' Expr FOR IDENTIFIER IN Expr IfBoolExpr ']'
     [java] predictive set is empty
     [java] Parser is successfully generated and written to "Table.java"
\end{lstlisting}
可以看到它和众多产生式冲突,并且预测集为空。具体而言,与数组常量的形式一致,都是\uline{[}开头,有左公因子,无法直接转化为LL(1)文法。
\section{}
将\uline{S1+/test1.decaf}的代码第六行int改成in:
\begin{lstlisting}[basicstyle=\ttfamily\small, numbers=left, frame=shadowbox, rulesepcolor=\color{red!20!green!20!blue!20},extendedchars=true,breaklines=true]
class Main {
    bool field;

    class Main another;

    in[] foo(int y) {
        int[] q;
        int i;

        q = new int[y];
        i = 0;
        while (i < y) {
            q[i] = i;
            i = i + 1;
        }

        return q;
    }

    static int main() {
        return this.foo(5).length();    
    }
}
\end{lstlisting}
报错如下:
\begin{lstlisting}[basicstyle=\ttfamily\small, numbers=left, frame=shadowbox, rulesepcolor=\color{red!20!green!20!blue!20},extendedchars=true,breaklines=true]
*** Error at (6,5): syntax error
*** Error at (6,19): syntax error
*** Error at (10,9): syntax error
*** Error at (10,21): syntax error
*** Error at (10,22): syntax error
*** Error at (11,9): syntax error
*** Error at (17,9): syntax error
\end{lstlisting}
理论上只要报第六行的error,实际上它报了其他错。因为并没有文法以\uline{in}开头,\uline{in}都是跟在\uline{identifier}之后。根据提供的错误处理方法,在处理到\uline{FieldList}的时候会报错,跳过第六行,7和8当做类中变量声明,10理应是函数实际上却是\uline{Expr}所以报错,之后同理。
\end{document}
